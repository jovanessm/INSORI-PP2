\chapter{Theoretische Grundlage der digitalen Souveränität}
\label{ch:theorie}
Der Begriff "Souveränität" bedeutet laut Stanford die höchste Autorität eines Gebiets \cite{sep-sovereignty}. Aber wenn man über die digitale Souveränität spricht, ist 
es schwierig zu sagen, welche Autorität auf welchem Gebiet in der Macht ist, weil Internet ein globale Netzwerk, in dem Daten aus verschiedene Quelle weltweit ausgetauscht
werden. Daher kann man sagen, dass die Regulierung des Internets über nationale Grenzen und damit über die staatliche Souveränität hinaus geht. Um die Definition der 
Digitalsouveränität besser zu verstehen, haben Stephane Couture und Sophie Toupin die Digitalsouveränität aus 5 Perspektiven analysiert, und zwar im Bezug auf Cyberspace, 
staatlich, einheimisch, sozial und personal \cite{Couture:2019}: Digitalsouveränität ist die Abhängigkeit, Kontrolle und Autonomie 
\begin{itemize}
\item eines Staats, einer Organization oder Individuelle, in der technologische Entwicklung etwas zu erfinden oder damit befassen und 
\item der Sicherheit und/oder Datenschutz der Individuelle oder Kollektive und in Bezug auf das Eigentum und die Kontrolle über Daten, die sich auf sich selbst, Bürger oder 
einen Staat beziehen. 
\end{itemize}
Außerdem kann auch die Digitalsouveränität in verschiedene Aspekte unterteilt \cite{BeyererMQuadeReussner2018}: 
\begin{itemize}
    \item Infrastruktursouveränität: technische Infrastrukturen vertrauenswürdig herzustellen, prüfen, und sie so zu betreiben, dass die darauf angebotenen Dienste 
    vertrauenswürdig sein können
    \item Datensouveränität: Fähigkeit, informiert und selbstbestimmt zu entscheiden, wie und von wem Informationen über die eigene Person oder Institution, eigene 
    Handlungen oder Produkte erhoben, verarbeitet und weitergegeben werden.
    \item Entscheidungssouveränität: die Möglichkeit, Ursprünge und Begründungen für Entscheidungen und Handlungsempfehlungen autonomer Systeme und Assistenten nachzuvollziehen
    und diese gegebenenfalls durch menschliches Eingreifen zu beeinflussen.
    \item Plattformsouveränität: die Marktmacht der großen Unternehmen durch Regulierung und bewusste Kundenentscheidungen auf ein Maß, in dem ein fairer Wettbewerb möglich bleibt
\end{itemize}


%
% Section: Infrastruktursouveränität
%
\section{Infrastruktursouveränität}
\label{sec:theorie:infra}
%
% Section: Datensouveränität
%
\section{Datensouveränität}
\label{sec:theorie:data}

%
% Section: Plattformsouveränität
%
\section{Platformsouveränität}
\label{sec:theorie:platform}

%
% Section: Entscheidungssouveränität
%
\section{Entscheidungssouveränität}
\label{sec:theorie:entscheidung}
