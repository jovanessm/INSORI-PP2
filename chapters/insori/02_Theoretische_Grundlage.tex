\chapter{Theoretische Grundlage der digitalen Souveränität}
\label{ch:theorie}
\section{Digitalsouveränität}
Der Begriff ``Souveränität'' bedeutet laut Stanford die höchste Autorität eines bestimmten Gebiets \cite{sep-sovereignty}. 
Überträgt man dieses Konzept auf den digitalen Raum, stellt sich die Frage, welche Autorität über welches Gebiet tatsächlich Macht ausüben kann.
Das Internet ist ein globale Netzwerk. 
Die Regulierung des Internets geht daher über die nationale Grenzen und damit über die traditionale staatliche Souveränität hinaus \cite{Barlow1996Independence}. 

Aber seit dem Aufkommen des Cloud Computings wird die Aussage widersprochen,
da die Cloudanbieter eine zentrale Rolle spielt, wie die Daten in dem Cloud gespeichert werden \cite{DeFilippiMcCarthy2012_CloudComputing}.
wie die Cloud funktioniert besprechen wir im nächsten Section.

Damit stellt sich die grundlegende Frage, was unter digitaler Souveränität überhaupt zu verstehen ist. 
Stephane Couture und Sophie Toupin haben die Digitalsouveränität aus 5 Perspektiven analysiert, und zwar im Bezug auf Cyberspace, 
staatlich, einheimisch, sozial und personal \cite{Couture:2019}: Digitalsouveränität ist die Abhängigkeit, Kontrolle und Autonomie 
\begin{itemize}
\item eines Staats, einer Organization oder Individuelle, in der technologische Entwicklung etwas zu erfinden oder damit befassen und 
\item der Sicherheit und/oder Datenschutz der Individuelle oder Kollektive und in Bezug auf das Eigentum und die Kontrolle über Daten, die sich auf sich selbst, Bürger oder 
einen Staat beziehen. 
\end{itemize}
Außerdem kann auch die Digitalsouveränität in verschiedene Aspekte unterteilt\cite{BeyererMQuadeReussner2018}: 
\begin{itemize}
    \item Infra\-struk\-tur\-sou\-ver\-ä\-ni\-tät: technische Infrastrukturen vertrauenswürdig herzustellen, 
    prüfen, und sie so zu betreiben, dass die darauf angebotenen Dienste 
    vertrauenswürdig sein können
    \item Datensouveränität: Fähigkeit, informiert und selbstbestimmt zu entscheiden, 
    wie und von wem Informationen über die eigene Person oder Institution, eigene 
    Handlungen oder Produkte erhoben, verarbeitet und weitergegeben werden.
    \item Entscheidungssouveränität: die Möglichkeit, Ursprünge und Begründungen für 
    Entscheidungen und Handlungsempfehlungen autonomer Systeme und Assistenten nachzuvollziehen
    und diese gegebenenfalls durch menschliches Eingreifen zu beeinflussen.
    \item Platt\-form\-sou\-ver\-ä\-ni\-tät: die Markt\-macht der gro\-ßen Unternehmen durch Regulierung und 
    bewusste Kundenentscheidungen auf ein Maß, in dem ein fairer Wettbewerb möglich bleibt
\end{itemize}

\section{Cloud Computing}
Vor der Ära von Cloud müssen Unternehmen bzw. Leute ihre Applikation in ihren eigenen Server (On-Premises) deployen. 
die Herausforderung, IT Infrastrukturen zu managen und provisionieren mit vorab genügend CAPEX (Capital Expenditures), 
wird durch die Erfindung der CloudComputing gelöst.

Cloud Computing bedeutet, dass die Datenverarbeitung nicht nur auf lokalen Computern erfolgt,
sondern in zentralisierten Einrichtungen durch externe Rechen- und Speicheranbieter \cite{magoules2013cloud}.
Es erlaubt allgegenwärtiger, bequemer und On-Demand Zugriff auf verteilte Rechenresourcen durch Netzwerk, 
die schnell zu provisionieren und zu entlassen \cite{mell2011nist}.
Amazon Web Service, der Vorreiter und aktuelle (2025) weltweit Marktführer bei Cloud-Diensten, 
hat im Jahr 2006 das erste Cloud-Angebot und zwar das Amazon Simple Storage Service (S3) und 
der Amazon Elastic Cloud Computing (EC2) auf den Markt gebracht \cite{AWS_OurOrigins}.

Es gibt mehrere Möglichkeit, wie man eine Cloud\-infrastuktur nutzt \cite{mell2011nist}.
Mit private Cloud werden die Infrastrukturen ausschließlich für eine Organization provisioniert und 
bei Public Cloud werden die Cloud-Infrastruktur für die offene Nutzung bereitgestellt. 
Hybrid Cloud ist der Fall, wo eine Organization die Flexibilität der Public Cloud nutzt, 
aber auch die erhöhte Sicherheit und Compliance der Private Cloud.
In dieser Arbeit wird der Umfang nur in Public Cloud eingeschränkt.

das Cloud Computing Modell kann man durch 5 Charakteristika erkennen \cite{mell2011nist}:
\begin{itemize}
    \item On-demand Self-Service: der Cloud Provider Kunde kann selber Rechenresourcen provisionieren wie 
    z.B Server oder Netzwerkspeicher, ohne der Bedarf von menschlichen Interaktion.
    \item Broad network access: die Cloud Services sind durch Netzwerk mit standartisierten Mechanismus erreichbar.
    \item Resource pooling: die Rechenresourcen sind konzentriert in einer Lage, um die Cloud Service Nutzer 
    zu bedienen durch multi-tenant Modell.
    \item Rapid elasticity: Rechenresourcen können elastisch und auch automatisch provisioniert und entlassen, 
    um die Ressourcen nach Bedarf zu skalieren.
    \item Measured Service: Services können durch Metriken gemessen werden, um die Ressourcen zu optimieren und Transparenz 
    an Kunden anzubieten.
\end{itemize}

\section{EU Regulierung zur Digitalsouveränität und Cloud}
