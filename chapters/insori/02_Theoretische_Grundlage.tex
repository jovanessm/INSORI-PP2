\chapter{Theoretische Grundlage der digitalen Souveränität}
\label{ch:theorie}
Der Begriff ``Souveränität'' bedeutet laut Stanford die höchste Autorität eines bestimmten Gebiets \cite{sep-sovereignty}. 
Überträgt man dieses Konzept auf den digitalen Raum, stellt sich die Frage, welche Autorität über welches Gebiet tatsächlich Macht ausüben kann.
Das Internet ist ein globale Netzwerk. 
Die Regulierung des Internets geht daher über die nationale Grenzen und damit über die traditionale staatliche Souveränität hinaus \cite{Barlow1996Independence}. 

Aber seit dem Aufkommen des Cloud Computings wird die Aussage widersprochen,
da die Cloudanbieter eine zentrale Rolle spielt, wie die Daten in dem Cloud gespeichert, verarbeitet und verteilt werden \cite{DeFilippiMcCarthy2012_CloudComputing}.
Um die Souveränität in der Cloud Gebiet zu bewerten, haben die EU Kommision einen Cloud Sovereignty Framework veröffentlicht \cite{EC_CloudSovereigntyFramework_2025}.

Damit stellt sich die grundlegende Frage, was unter digitaler Souveränität überhaupt zu verstehen ist. 
Stephane Couture und Sophie Toupin haben die Digitalsouveränität aus 5 Perspektiven analysiert, und zwar im Bezug auf Cyberspace, 
staatlich, einheimisch, sozial und personal \cite{Couture:2019}: Digitalsouveränität ist die Abhängigkeit, Kontrolle und Autonomie 
\begin{itemize}
\item eines Staats, einer Organization oder Individuelle, in der technologische Entwicklung etwas zu erfinden oder damit befassen und 
\item der Sicherheit und/oder Datenschutz der Individuelle oder Kollektive und in Bezug auf das Eigentum und die Kontrolle über Daten, die sich auf sich selbst, Bürger oder 
einen Staat beziehen. 
\end{itemize}
Außerdem kann auch die Digitalsouveränität in verschiedene Aspekte unterteilt\cite{BeyererMQuadeReussner2018}: 
\begin{itemize}
    \item Infrastruktursouveränität: technische Infrastrukturen vertrauenswürdig herzustellen, prüfen, und sie so zu betreiben, dass die darauf angebotenen Dienste 
    vertrauenswürdig sein können
    \item Datensouveränität: Fähigkeit, informiert und selbstbestimmt zu entscheiden, wie und von wem Informationen über die eigene Person oder Institution, eigene 
    Handlungen oder Produkte erhoben, verarbeitet und weitergegeben werden.
    \item Entscheidungssouveränität: die Möglichkeit, Ursprünge und Begründungen für Entscheidungen und Handlungsempfehlungen autonomer Systeme und Assistenten nachzuvollziehen
    und diese gegebenenfalls durch menschliches Eingreifen zu beeinflussen.
    \item Plattformsouveränität: die Marktmacht der großen Unternehmen durch Regulierung und bewusste Kundenentscheidungen auf ein Maß, in dem ein fairer Wettbewerb möglich bleibt
\end{itemize}
